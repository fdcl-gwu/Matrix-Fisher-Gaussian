\documentclass[10pt]{article}

\usepackage{amssymb,amsmath,amsthm}
\usepackage{bm}
\usepackage{graphicx,subcaption}
\usepackage[letterpaper, top=1in, left=1in, right=1in, bottom=1in]{geometry}

\newtheorem{definition}{Definition}
\newtheorem{theorem}{Theorem}
\newtheorem{lemma}{Lemma}
\newtheorem{remark}{Remark}

\title{\vspace{-4ex}\textbf{Less Sigma Points\vspace{-4ex}}}
\date{}

\graphicspath{{./figs/}}

\newcommand{\norm}[1]{\ensuremath{\left\| #1 \right\|}}
\newcommand{\fnorm}[1]{\ensuremath{\left\| #1 \right\|_\mathrm{F}}}
\newcommand{\tr}[1]{\ensuremath{\mathrm{tr}\left( #1 \right)}}
\newcommand{\etr}[1]{\ensuremath{\mathrm{etr}\left( #1 \right)}}
\newcommand{\expect}[1]{\ensuremath{\mathrm{E}\left[ #1 \right]}}
\newcommand{\SO}{\ensuremath{\mathrm{SO(3)}}}
\newcommand{\real}[1]{\ensuremath{\mathbb{R}^{ #1 }}}
\newcommand{\diff}[1]{\ensuremath{\mathrm{d} #1}}

\begin{document}

\maketitle

\section{Adding a Gaussian noise}
Suppose $(R,x)$ follow Matrix Fisher-Gaussian distribution with parameters $(\mu,\Sigma,P,U,S,V)$, we want to compute the distribution of $(R,x+W)$, where $W\in\real{n}$ follows $\mathcal{N}(\mu_w,\Sigma_w)$, and is independent of $(R,x)$.
\begin{lemma} \label{lamma:AddGaussianNoise}
	$(R,x+W)\sim\mathcal{MG}(\mu+\mu_w,\Sigma+\Sigma_w,P,U,S,V)$.
\end{lemma}
\begin{proof}
	Let $y=x+W$, then the density function for $(R,y)$ is
	\begin{align} \label{eqn:fRy}
		f_{R,y}(R,y) &= \int_{x\in\real{n}}f_{R,x}(R,x)f_W(y-x)\diff{x} \nonumber \\
		&= \frac{1}{c_{R,x}c_w}\int_{x\in\real{n}} \etr{FR^T} \exp\left(-\frac{1}{2}(x-\mu_c)^T\Sigma_c^{-1}(x-\mu_c)\right) \nonumber \\ 
		&\qquad\qquad\times \exp\left(-\frac{1}{2}(y-x-\mu_w)^T\Sigma_w^{-1}(y-x-\mu_w)\right) \diff{x}
	\end{align}
	By the convolution theorem for Gaussian distribution, this density function can be simplified as
	\begin{equation*}
		f_{R,y}(R,y) = \frac{1}{c_{R,x}c_w}\etr{FR^T} \exp\left\{-\frac{1}{2}(y-\mu_c-\mu_w)^T(\Sigma_c+\Sigma_w)^{-1}(y-\mu_c-\mu_w)\right\}
	\end{equation*}
	Because $\mu_c+\mu_w = \mu+\mu_w+P\nu_R$, and $\Sigma_c+\Sigma_w = \Sigma+\Sigma_w+P(\tr{S}I_{3\times 3}-S)P^T$, it follows that $(R,x+W)\sim\mathcal{MG}(\mu+\mu_w,\Sigma+\Sigma_w,P,U,S,V)$.
\end{proof}

\section{Reducing sigma points}

Consider the following discrete stochastic process
\begin{align}
	R(t+\Delta t) &= R(t)\exp\left\{ \Delta t (\Omega(t)+x(t))^\wedge + (H_1\Delta W_1)^\wedge \right\} \\
	x(t+\Delta t) &= x(t) + H_2\Delta W_2
\end{align}
Suppose at time $t$, $(R(t),x(t))$ follows Matrix Fisher-Gaussian distribution with parameters $(\mu,\Sigma,P,U,S,V)$, we want to calculate the distribution for $(R(t+\Delta t),x(t+\Delta t))$.
By Lemma \ref{lamma:AddGaussianNoise}, we can first calculate the distribution for $(R(t+\Delta t),x(t))$, and fit it to a Matrix Fisher-Gaussian distribution, then add a Gaussian noise to the linear part.

Because $x(t)$ is not changed in $(R(t+\Delta t),x(t))$, it's moments are also unchanged.
We still need to calculate $\expect{R(t+\Delta t)}$, $\expect{\nu_R(t+\Delta t)}$, $\expect{x(t)\nu_R^T(t+\Delta t)}$, and $\expect{\nu_R(t+\Delta t)\nu_R^T(t+\Delta t)}$.
These moments can be calculated using law of total expectation, e.g.
\begin{equation} \label{eqn:totExpect}
	\expect{R(t+\Delta t)} = \expect{\expect{R(t+\Delta t)\left|\right.x(t)}}.
\end{equation}
We propose to approximate the marginal distribution of $x(t)$ using sigma points, i.e.
\begin{equation}
	f_x(x) = \sum_i w_i\delta(x-x_i)
\end{equation}
With this, the expectation in \eqref{eqn:totExpect} becomes
\begin{equation}
	\expect{R(t+\Delta t)} = \sum_i w_i\expect{R(t+\Delta t)\left|\right. x_i},
\end{equation}
where the conditional expectation is given by
\begin{equation}
	\expect{R(t+\Delta t)|x_i} = \expect{R(t)|x_i}\exp\left(I_{3\times 3}+\frac{\Delta t}{2}(-\tr{G}I_{3\times 3}+G)+O(\Delta t^{1.5})\right)\exp\left((\hat{\Omega}(t)+\hat{x}_i)\Delta t\right).
\end{equation}
From the expectation of $R(t+\Delta t)$, the new $U'$, $S'$ and $V'$ at $t+\Delta t$ can be calculated using the usual technique.
Thus, the new $\nu_R(t+\Delta) = \left(U'^TR(t+\Delta t)V'S'-S'V'^TR(t+\Delta t)^TU'\right)^\vee$, and we have the following moments
\begin{align}
	\expect{\nu_R(t+\Delta)} &= \left( U'^T\expect{R(t+\Delta t)}V'S' - S'V'^T\expect{R(t+\Delta t)}^TU' \right)^\vee \\
	\expect{x\nu_R^T(t+\Delta t)} &= \sum_i w_ix_i\left[\left( U'^T\expect{R(t+\Delta t)|x_i}V'S' - S'V'^T\expect{R(t+\Delta t)|x_i}^TU' \right)^\vee\right]^T
\end{align}

Next, let us calculate $\expect{\nu_R(t+\Delta t)\nu_R^T(t+\Delta t)}$. Before doing this, we need some messy calculations.
\begin{lemma}
	If $R$ follows Matrix Fisher distribution with the parameter $F=USV^T$, then its second order moments are given by
	\begin{align}
		\expect{R_{ij}R_{kl}} &= \sum_{m=1}^{3}\sum_{n=1}^{3}U_{im}U_{kn}V_{jm}V_{ln}\expect{Q_{mm}Q_{nn}} \nonumber \\
		&+ \sum_{m=1}^{3}\sum_{\substack{n=1\\n\neq m}}^{3}U_{im}U_{km}V_{jn}V_{ln}\expect{Q_{nm}^2} \nonumber \\
		&+ \sum_{m=1}^{3}\sum_{\substack{n=1\\n\neq m}}^{3}U_{im}U_{kn}V_{jn}V_{lm}\expect{Q_{nm}Q_{mn}}.
	\end{align}
\end{lemma}
\begin{proof}
	First write $R$ as $UQV^T$, so $R_{ij} = \sum_{m=1}^{3}\sum_{n=1}^{3}U_{im}V_{jn}Q_{mn}$.
	Then we get
	\begin{equation*}
		\expect{R_{ij}R_{kl}} = \sum_{\alpha=1}^{3}\sum_{\beta=1}^{3}\sum_{m=1}^{3}\sum_{n=1}^{3}U_{im}V_{jn}U_{k\alpha}V_{l\beta}\expect{Q_{mn}Q_{\alpha\beta}}.
	\end{equation*}
\end{proof}

\appendix
\section{A direct calculation for Lemma 1}
This calculation is for future reference, for example, when a similar calculation is needed for some other distribution, comparing with Gaussian case is sometimes motivating.
First note that for any two invertible matrices $A$, $B$, if $A+B$ and $A^{-1}+B^{-1}$ are also invertible, then $(A^{-1}+B^{-1})^{-1} = A-A(A+B)^{-1}A = B-B(A+B)^{-1}B$.

The exponentiated expression in \eqref{eqn:fRy} can be written as
\begin{align*}
	&-\frac{1}{2}x^T\Sigma_c^{-1}x + x^T\Sigma_c^{-1}\mu_c - \frac{1}{2}\mu_c^T\Sigma_c^{-1}\mu_c - \frac{1}{2}x^T\Sigma_w^{-1}x + x^T\Sigma_w(\mu_w-y) - \frac{1}{2}(\mu_w-y)^T\Sigma_w^{-1}(\mu_w-y) \\
	= &-\frac{1}{2}x^T(\Sigma_c^{-1}+\Sigma_w^{-1})x + x^T(\Sigma_c^{-1}+\Sigma_w^{-1})(\Sigma_c^{-1}+\Sigma_w^{-1})^{-1}\Sigma_c^{-1}\mu_c + x^T(\Sigma_c^{-1}+\Sigma_w^{-1})(\Sigma_c^{-1}+\Sigma_w^{-1})^{-1}\Sigma_w^{-1}(\mu_w-y) \\
	&-\frac{1}{2}\left[(\Sigma_c^{-1}+\Sigma_w^{-1})^{-1}\Sigma_c^{-1}\mu_c+(\Sigma_c^{-1}+\Sigma_w^{-1})^{-1}\Sigma_w^{-1}(\mu_w-y)\right]^T(\Sigma_c^{-1}+\Sigma_w^{-1}) \\
	&\qquad\cdot\left[(\Sigma_c^{-1}+\Sigma_w^{-1})^{-1}\Sigma_c^{-1}\mu_c+(\Sigma_c^{-1}+\Sigma_w^{-1})^{-1}\Sigma_w^{-1}(\mu_w-y)\right] \\
	&+\frac{1}{2}\left[(\Sigma_c^{-1}+\Sigma_w^{-1})^{-1}\Sigma_c^{-1}\mu_c+(\Sigma_c^{-1}+\Sigma_w^{-1})^{-1}\Sigma_w^{-1}(\mu_w-y)\right]^T(\Sigma_c^{-1}+\Sigma_w^{-1}) \\
	&\qquad\cdot\left[(\Sigma_c^{-1}+\Sigma_w^{-1})^{-1}\Sigma_c^{-1}\mu_c+(\Sigma_c^{-1}+\Sigma_w^{-1})^{-1}\Sigma_w^{-1}(\mu_w-y)\right] - \frac{1}{2}\mu_c^T\Sigma_w^{-1}\mu_c - \frac{1}{2}(\mu_w-y)^T\Sigma_w^{-1}(\mu_w-y) \\
	= &-\frac{1}{2}\left[x-(\Sigma_c^{-1}+\Sigma_w^{-1})^{-1}\Sigma_c^{-1}\mu_c-(\Sigma_c^{-1}+\Sigma_w^{-1})^{-1}\Sigma_w^{-1}(\mu_w-y)\right]^T(\Sigma_c^{-1}+\Sigma_w^{-1}) \\
	&\qquad\cdot\left[x-(\Sigma_c^{-1}+\Sigma_w^{-1})^{-1}\Sigma_c^{-1}\mu_c-(\Sigma_c^{-1}+\Sigma_w^{-1})^{-1}\Sigma_w^{-1}(\mu_w-y)\right] \\
	&+\frac{1}{2}\left[(\Sigma_c^{-1}+\Sigma_w^{-1})^{-1}\Sigma_c^{-1}\mu_c+(\Sigma_c^{-1}+\Sigma_w^{-1})^{-1}\Sigma_w^{-1}(\mu_w-y)\right]^T(\Sigma_c^{-1}+\Sigma_w^{-1}) \\
	&\qquad\cdot\left[(\Sigma_c^{-1}+\Sigma_w^{-1})^{-1}\Sigma_c^{-1}\mu_c+(\Sigma_c^{-1}+\Sigma_w^{-1})^{-1}\Sigma_w^{-1}(\mu_w-y)\right] - \frac{1}{2}\mu_c^T\Sigma_w^{-1}\mu_c - \frac{1}{2}(\mu_w-y)^T\Sigma_w^{-1}(\mu_w-y)
\end{align*}
In the last expression, all of terms but the first do not depend on $x$, therefore can be moved out of the integral.
The first term is a Gaussian density which integrates to $1$, so after integration, the remaining terms are (using the equations in the first paragraph at the second equation):
\begin{align*}
	&\frac{1}{2}\left[(\Sigma_c^{-1}+\Sigma_w^{-1})^{-1}\Sigma_c^{-1}\mu_c+(\Sigma_c^{-1}+\Sigma_w^{-1})^{-1}\Sigma_w^{-1}(\mu_w-y)\right]^T(\Sigma_c^{-1}+\Sigma_w^{-1}) \\
	&\qquad\cdot\left[(\Sigma_c^{-1}+\Sigma_w^{-1})^{-1}\Sigma_c^{-1}\mu_c+(\Sigma_c^{-1}+\Sigma_w^{-1})^{-1}\Sigma_w^{-1}(\mu_w-y)\right] - \frac{1}{2}\mu_c^T\Sigma_w^{-1}\mu_c - \frac{1}{2}(\mu_w-y)^T\Sigma_w^{-1}(\mu_w-y) \\
	= &\frac{1}{2}\mu_c^T\Sigma_c^{-1}(\Sigma_c^{-1}+\Sigma_w^{-1})^{-1}\Sigma_c^{-1}\mu_c + \frac{1}{2}(\mu_w-y)^T\Sigma_w^{-1}(\Sigma_c^{-1}+\Sigma_w^{-1})^{-1}\Sigma_w^{-1}(\mu_w-y) \\
	&+\mu_c^T\Sigma_c^{-1}(\Sigma_c^{-1}+\Sigma_w^{-1})^{-1}\Sigma_w^{-1}(\mu_w-y) - \frac{1}{2}\mu_c^T\Sigma_w^{-1}\mu_c - \frac{1}{2}(\mu_w-y)^T\Sigma_w^{-1}(\mu_w-y) \\
	= &-\frac{1}{2}\mu_c^T(\Sigma_c+\Sigma_w)^{-1}\mu_c - \frac{1}{2}(\mu_w-y)^T(\Sigma_c+\Sigma_w)^{-1}(\mu_w-y) + \mu_c^T(I-(\Sigma_c+\Sigma_w)^{-1}\Sigma_c)\Sigma_w^{-1}(\mu_w-y) \\ 
	= &-\frac{1}{2}\mu_c^T(\Sigma_c+\Sigma_w)^{-1}\mu_c - \frac{1}{2}(\mu_w-y)^T(\Sigma_c+\Sigma_w)^{-1}(\mu_w-y) + \mu_c^T(\Sigma_c+\Sigma_w)^{-1}(\mu_w-y) \\
	= &-\frac{1}{2}(\mu_c+\mu_w-y)^T(\Sigma_c+\Sigma_w)^{-1}(\mu_c+\mu_w-y).
\end{align*}
This is the needed result.

\end{document}

