\documentclass[10pt]{article}

\usepackage{amssymb,amsmath,amsthm}
\usepackage{bm}
\usepackage{graphicx,subcaption}
\usepackage[letterpaper, top=1in, left=1in, right=1in, bottom=1in]{geometry}

\newtheorem{definition}{Definition}
\newtheorem{proposition}{Proposition}
\newtheorem{theorem}{Theorem}
\newtheorem{lemma}{Lemma}
\newtheorem{remark}{Remark}

\title{\vspace{-4ex}\textbf{Third order moment for Matrix Fisher distribution\vspace{-4ex}}}
\date{}

\newcommand{\norm}[1]{\ensuremath{\left\| #1 \right\|}}
\newcommand{\fnorm}[1]{\ensuremath{\left\| #1 \right\|_\mathrm{F}}}
\newcommand{\tr}[1]{\ensuremath{\mathrm{tr}\left( #1 \right)}}
\newcommand{\etr}[1]{\ensuremath{\mathrm{etr}\left\{ #1 \right\}}}
\newcommand{\expect}[1]{\ensuremath{\mathrm{E}\left[ #1 \right]}}
\newcommand{\SO}{\ensuremath{\mathrm{SO(3)}}}
\newcommand{\real}[1]{\ensuremath{\mathbb{R}^{ #1 }}}
\newcommand{\diff}[1]{\ensuremath{\mathrm{d} #1}}

\begin{document}

\maketitle

\section{General formula}

\begin{align}
	\frac{\partial^3 M_Q(T)}{\partial T_{ij} \partial T_{kl} \partial T_{mn}} = &\frac{1}{c(S)}\left.\left( \sum_{\alpha=1}^{3}\sum_{\beta=1}^{3}\sum_{\gamma=1}^{3} \frac{\partial^3 c(S+T)}{\partial s_\alpha \partial s_\beta \partial s_\gamma} \frac{\partial s_\alpha}{\partial T_{ij}} \frac{\partial s_\beta}{\partial T_{kl}} \frac{\partial s_\gamma}{\partial T_{mn}} \right)\right|_{T=0} \nonumber \\ 
	&+ \frac{1}{c(S)} \left.\left( \sum_{\alpha=1}^3\sum_{\beta=1}^3 \frac{\partial^2 c(S+T)}{\partial s_\alpha \partial s_\beta } \left[ \frac{\partial^2 s_\alpha}{\partial T_{ij} \partial T_{kl}} \frac{\partial s_\beta}{\partial T_{mn}} + \frac{\partial^2 s_\alpha}{\partial T_{ij} \partial T_{mn}} \frac{\partial s_\beta}{\partial T_{kl}} + \frac{\partial^2 s_\alpha}{\partial T_{kl} \partial T_{mn}} \frac{\partial s_\beta}{\partial T_{ij}} \right] \right)\right|_{T=0} \nonumber \\
	&+\frac{1}{c(S)} \left.\left( \sum_{\alpha=1} \frac{\partial c(S+T)}{\partial s_\alpha}\frac{\partial^3 s_\alpha}{\partial T_{ij} \partial T_{kl} \partial T_{mn}} \right)\right|_{T=0}
\end{align}

\section{$\expect{Q_{ii}Q_{jj}Q_{kk}}$}

\begin{equation}
	\frac{\partial^3 M_Q(T)}{\partial T_{ii} \partial T_{jj} \partial T_{kk}} = \frac{1}{c(S)} \left. \frac{\partial^3 c(S)}{\partial s_i \partial s_j \partial s_k} \right|_{T=0}
\end{equation}

\section{$\expect{Q_{ii}Q_{jk}Q_{jk}}$ ($j\neq k$)}

Since $\left.\frac{\partial s_\alpha}{\partial T_{ij}}\right|_{T=0} = \left.U_{i\alpha}V_{j\alpha}\right|_{T=0} = \delta_{i\alpha}\delta_{j\alpha}$, so
\begin{equation}
	\frac{\partial^3 M_Q(T)}{\partial T_{ii} \partial^2 T_{jk}} = \frac{1}{c(S)}\left.\left( \sum_{\alpha=1}^3\sum_{\beta=1}^3 \frac{\partial^2 c(S+T)}{\partial s_\alpha \partial s_\beta} \frac{\partial^2 s_\alpha}{\partial T_{jk}^2} \frac{\partial s_\beta}{\partial T_{ii}} \right)\right|_{T=0} + \frac{1}{c(S)}\left.\left( \sum_{\alpha=1}^3 \frac{\partial c(S+T)}{\partial s_\alpha} \frac{\partial^3 s_\alpha}{\partial T_{ii}\partial T^2_{jk}} \right)\right|_{T=0}
\end{equation}
The first term can be simplified as
\begin{align}
	\frac{1}{c(S)}\left.\left( \sum_{\alpha=1}^3\sum_{\beta=1}^3 \frac{\partial^2 c(S+T)}{\partial s_\alpha \partial s_\beta} \frac{\partial^2 s_\alpha}{\partial T_{jk}^2} \frac{\partial s_\beta}{\partial T_{ii}} \right)\right|_{T=0} &= \frac{1}{c(S)}\left.\left( \sum_{\alpha=1}^3 \frac{\partial^2 c(S+T)}{\partial s_\alpha \partial s_i} \frac{\partial^2 s_\alpha}{\partial T_{jk}^2} \right)\right|_{T=0} \nonumber \\
	&= \frac{1}{c(S)}\left.\left( \sum_{\alpha=1}^3 \frac{\partial^2 c(S+T)}{\partial s_\alpha \partial s_i} \left[ V_{k\alpha}\frac{\partial U_{j\alpha}}{\partial T_{jk}} + U_{j\alpha}\frac{\partial V_{k\alpha}}{\partial T_{jk}} \right] \right)\right|_{T=0} \nonumber \\
	&= \frac{1}{c(S)} \left.\left( \frac{\partial^2 c(S+T)}{\partial s_j \partial s_i} \frac{\partial V_{kj}}{\partial T_{jk}} + \frac{\partial^2 c(S+T)}{\partial s_k \partial s_i} \frac{\partial U_{jk}}{\partial T_{jk}} \right)\right|_{T=0} \nonumber \\
	&= \frac{1}{c(S)} \left( \frac{\partial^2 c(S)}{\partial s_i \partial s_k} \frac{s_k}{s_k^2-s_j^2} - \frac{\partial^2 c(S)}{\partial s_i \partial s_j} \frac{s_j}{s_k^2-s_j^2} \right)
\end{align}
Note that $\left.\frac{\partial U}{\partial T_{ii}}\right|_{T=0} = \left.\frac{\partial V}{\partial T_{ii}}\right|_{T=0} = 0$, the second term can be simplified as 
\begin{align} \label{eqn:22}
	\frac{1}{c(S)}\left.\left( \sum_{\alpha=1}^3 \frac{\partial c(S+T)}{\partial s_\alpha} \frac{\partial^3 s_\alpha}{\partial T_{ii}\partial T^2_{jk}} \right)\right|_{T=0} &= \frac{1}{c(S)}\left.\left( \sum_{\alpha=1}^3 \frac{\partial c(S+T)}{\partial s_\alpha} \frac{\partial}{\partial T_{ii}} \left[ V_{k\alpha}\frac{\partial U_{j\alpha}}{\partial T_{jk}} + U_{j\alpha}\frac{\partial V_{k\alpha}}{\partial T_{jk}} \right] \right)\right|_{T=0} \nonumber \\
	&= \frac{1}{c(S)}\left.\left( \sum_{\alpha=1}^3 \frac{\partial c(S+T)}{\partial s_\alpha} \left[ V_{k\alpha}\frac{\partial^2 U_{j\alpha}}{\partial T_{jk} \partial T_{ii}} + U_{j\alpha}\frac{\partial^2 V_{k\alpha}}{\partial T_{jk} \partial T_{ii}} \right] \right)\right|_{T=0} \nonumber \\
	&= \frac{1}{c(s)} \left.\left( \frac{\partial c(S+T)}{\partial s_k}\frac{\partial^2 U_{jk}}{\partial T_{jk} \partial T_{ii}} + \frac{\partial c(S+T)}{\partial s_j} \frac{\partial^2 V_{kj}}{\partial T_{jk} \partial T_{ii}} \right)\right|_{T=0}
\end{align}

Recall that $\frac{\partial U}{\partial T_{ij}} = U\Omega_U^{ij}$, $\frac{\partial V}{\partial T_{ij}} = -V\Omega_V^{ij}$, where $\Omega_U^{ij}$ and $\Omega_V^{ij}$ are anti-symmetric matrices with non-diagonal elements given by the solutions of
\begin{align} \label{eqn:Omega}
	s_l\Omega_{U_{kl}}^{ij} + s_k\Omega_{V_{kl}}^{ij} &= U_{ik}V_{jl} \nonumber \\
	s_k\Omega_{U_{kl}}^{ij} + s_l\Omega_{V_{kl}}^{ij} &= -U_{il}V_{jk}.
\end{align}
With this, let $(j,k,m)\in\mathcal{I}$, the two terms in \eqref{eqn:22} becomes
\begin{align}
	\left. \frac{\partial^2 U_{jk}}{\partial T_{jk} \partial T_{ii}} \right|_{T=0} &= \frac{\partial}{\partial T_{ii}} \left.\left( U_{jj}\Omega_{U_{jk}}^{jk} + U_{jm}\Omega_{U_{mk}}^{jk} \right)\right|_{T=0} = \left.\frac{\partial \Omega_{U_{jk}}^{jk}}{\partial T_{ii}}\right|_{T=0} \nonumber \\
	\left. \frac{\partial^2 V_{kj}}{\partial T_{jk} \partial T_{ii}} \right|_{T=0} &= -\frac{\partial}{\partial T_{ii}} \left.\left( V_{kk}\Omega_{V_{kj}}^{jk} + V_{km}\Omega_{V_{mj}}^{jk} \right)\right|_{T=0} = -\left.\frac{\partial \Omega_{V_{kj}}^{jk}}{\partial T_{ii}}\right|_{T=0} = \left.\frac{\partial \Omega_{V_{jk}}^{jk}}{\partial T_{ii}}\right|_{T=0}
\end{align}
Differentiate both sides of \eqref{eqn:Omega} with respect to $T_{ii}$ and use proper indices, we get
\begin{align} \label{eqn:dUVdT}
	\frac{\partial s_k}{\partial T_{ii}}\Omega_{U_{jk}}^{jk} +  s_k\frac{\partial \Omega_{U_{jk}}^{jk}}{\partial T_{ii}} + \frac{\partial s_j}{\partial T_{ii}}\Omega_{V_{jk}}^{jk} + s_j\frac{\partial \Omega_{V_{jk}}^{jk}}{\partial T_{ii}} &= U_{jj}\frac{\partial V_{kk}}{\partial T_{ii}} + V_{kk}\frac{\partial U_{jj}}{\partial T_{ii}} \nonumber \\
	\frac{\partial s_j}{\partial T_{ii}}\Omega_{U_{jk}}^{jk} +  s_j\frac{\partial \Omega_{U_{jk}}^{jk}}{\partial T_{ii}} + \frac{\partial s_k}{\partial T_{ii}}\Omega_{V_{jk}}^{jk} + s_k\frac{\partial \Omega_{V_{jk}}^{jk}}{\partial T_{ii}} &= -U_{jk}\frac{\partial V_{kj}}{\partial T_{ii}} - V_{kj}\frac{\partial U_{jk}}{\partial T_{ii}}
\end{align}
Now consider two conditions: (a) $i\neq j$ and $i\neq k$.
Evaluate the above equations at $T=0$, we get
\begin{align}
	\left.\frac{\partial \Omega_{U_{jk}}^{jk}}{\partial T_{ii}}\right|_{T=0} &= \left.\frac{\partial \Omega_{V_{jk}}^{jk}}{\partial T_{ii}}\right|_{T=0} = 0 
\end{align}
Combining all these results, the third order moments of the form in this subsection is given by
\begin{align}
	\expect{Q_{ii}Q_{jk}Q_{jk}} &= \frac{1}{c(S)} \left( \frac{\partial^2 c(S)}{\partial s_i \partial s_k} \frac{s_k}{s_k^2-s_j^2} - \frac{\partial^2 c(S)}{\partial s_i \partial s_j} \frac{s_j}{s_k^2-s_j^2} \right) \nonumber \\
	\expect{Q_{ii}Q_{jk}Q_{kj}} &= \frac{1}{c(S)} \left( \frac{\partial^2 c(S)}{\partial s_i \partial s_k} \frac{s_j}{s_k^2-s_j^2} - \frac{\partial^2 c(S)}{\partial s_i \partial s_j} \frac{s_k}{s_k^2-s_j^2} \right)
\end{align}
For the second condition (b) $i=j\neq k$, solving \eqref{eqn:Omega} gives:
\begin{align}
	\Omega_{U_{jk}}^{jk} &= \frac{s_kU_{jj}V_{kk}+s_jU_{jk}V_{kj}}{s_k^2-s_j^2} \nonumber \\
	\Omega_{V_{jk}}^{jk} &= \frac{s_jU_{jj}V_{kk}+s_kU_{jk}V_{jk}}{s_j^2-s_k^2}
\end{align}
\eqref{eqn:dUVdT} gives
\begin{align}
	\left.\frac{\partial \Omega_{U_{jk}}^{jk}}{\partial T_{ii}}\right|_{T=0} &= \left.\frac{s_j\Omega_{U_{jk}}^{jk} - s_k\Omega_{V_{jk}}^{jk}}{s_j^2-s_k^2}\right|_{T=0} = \frac{2s_js_k}{(s_j^2-s_k^2)^2} \nonumber \\
	\left.\frac{\partial \Omega_{V_{jk}}^{jk}}{\partial T_{ii}}\right|_{T=0} &= \left.\frac{s_k\Omega_{U_{jk}}^{jk} - s_j\Omega_{V_{jk}}^{jk}}{s_j^2-s_k^2}\right|_{T=0} = -\frac{s_j^2+s_k^2}{(s_j^2-s_k^2)^2}
\end{align}
Combining these results, we have
\begin{align}
	\expect{Q_{jj}Q_{jk}Q_{jk}} &= \frac{1}{c(S)} \left( \frac{\partial^2 c(S)}{\partial s_j \partial s_k} \frac{s_k}{s_k^2-s_j^2} - \frac{\partial^2 c(S)}{\partial s_j^2} \frac{s_j}{s_k^2-s_j^2} - \frac{\partial c(S)}{\partial s_j} \frac{s_j^2+s_k^2}{(s_j^2-s_k^2)^2} + \frac{\partial c(S)}{\partial s_k} \frac{2s_js_k}{(s_j^2-s_k^2)^2} \right)
\end{align}

\section{$\expect{Q_{ij}Q_{jk}Q_{ki}}$ ($i\neq j\neq k$)}
\begin{equation*}
	\frac{\partial^3 M_Q(T)}{\partial T_{ij} \partial T_{jk} \partial T_{ki}} = \frac{1}{c(S)}\left.\left( \sum_{\alpha=1}^3 \frac{\partial c(S+T)}{\partial s_\alpha} \frac{\partial^3 s_\alpha}{\partial T_{ij} \partial T_{jk} \partial T_{ki}} \right)\right|_{T=0}
\end{equation*}

\end{document}

