\documentclass[10pt]{article}

\usepackage[letterpaper,margin=0.5in]{geometry}

\usepackage{amssymb,amsmath,amsthm}
\usepackage{bm}
\usepackage{graphicx,subcaption}
\usepackage{xcolor}

\newtheorem{definition}{Definition}
\newtheorem{proposition}{Proposition}
\newtheorem{theorem}{Theorem}
\newtheorem{lemma}{Lemma}
\newtheorem{remark}{Remark}

\title{\vspace{-4ex}\textbf{IMU Pre-integration\vspace{-4ex}}}
\date{}

\newcommand{\norm}[1]{\ensuremath{\left\| #1 \right\|}}
\newcommand{\fnorm}[1]{\ensuremath{\left\| #1 \right\|_\mathrm{F}}}
\newcommand{\tr}[1]{\ensuremath{\mathrm{tr}\left( #1 \right)}}
\newcommand{\etr}[1]{\ensuremath{\mathrm{etr}\left\{ #1 \right\}}}
\newcommand{\expect}[1]{\ensuremath{\mathrm{E}\left[ #1 \right]}}
\newcommand{\SO}{\ensuremath{\mathrm{SO(3)}}}
\newcommand{\real}[1]{\ensuremath{\mathbb{R}^{ #1 }}}
\newcommand{\diff}[1]{\ensuremath{\mathrm{d} #1}}
\newcommand{\expb}[1]{\ensuremath{\mathrm{exp}\left\{#1\right\}}}

\begin{document}

\maketitle

Let the continuous kinematics model of an IMU be
\begin{align}
	R^T\diff{R} &= \hat{\Omega}\diff{t} + H_g\diff{W}_t^g, \\
	\diff{v} &= (Ra + g)\diff{t} + R(H_a\diff{W}_t^a), \\
	\diff{p} &= v\diff{t}.
\end{align}
The corresponding discrete time kinematics is
\begin{align}
	R_{i+1} &= R_i\exp\left( h\hat{\Omega}_i + (H_g\Delta W_t^g)^\wedge \right), \\
	v_{i+1} &= v_i + hg + R_i(ha_i + H_a\Delta W_t^a), \\
	p_{i+1} &= p_i + hv_i + \tfrac{1}{2}h^2g + \tfrac{1}{2}R_i\left(h^2a_i + H_a\int_0^h W_t^a\diff{t}\right).
\end{align}
Let $i$, $j$ be the indices of two key frames, between which there are $(j-i)$ IMU.
Define the following increments for all $k=i+1,\ldots,j$:
\begin{gather}
	\Delta R_{ik} = R_i^TR_k = \prod_{l=i}^{k-1} \exp\left( h\hat{\Omega}_l + (H_g\Delta W_t^g)^\wedge \right), \label{eqn:Delta R} \\
	\Delta v_{ik} = R_i^T(v_k-v_i-(k-i)hg) = \sum_{l=i}^{k-1} \Delta R_{il}(ha_l + H_a\Delta W_t^a). \label{eqn:Delta v}
\end{gather}
The expressions for $\Delta p_{ik}$ is trickier, as we want it to be independent of $R_i$, $v_i$ and $p_i$.
\begin{align*}
	p_k &= p_i + \sum_{l=i}^{k-1} \left[ hv_l + \tfrac{1}{2}h^2g + \tfrac{1}{2}R_l \left( h^2a_l + H_a\int_0^h W_t^a\diff{t} \right) \right] \\
	&= p_i + \tfrac{1}{2}(k-i)h^2g + \sum_{l=i}^{k-1} \left[ h(v_l-v_i-(l-i)hg) + \tfrac{1}{2}R_l \left( h^2a_l + H_a\int_0^h W_t^a\diff{t} \right) \right] + \sum_{l=i}^{k-1} \left[ h(v_i+(l-i)hg) \right] \\
	&= p_i + (k-i)hv_i + \tfrac{1}{2}(k-i)^2h^2g + \sum_{l=i}^{k-1} \left[ h(v_l-v_i-(l-i)hg) + \tfrac{1}{2}R_l \left( h^2a_l + H_a\int_0^h W_t^a\diff{t} \right) \right].
\end{align*}
This indicates
\begin{align} \label{eqn: Delta p}
	\Delta p_{ik} = R_i^T\left( p_k - p_i - (k-i)hv_i - \tfrac{1}{2}(k-i)^2h^2g \right) = \sum_{l=i}^{k-1} \left[ h\Delta v_{il} + \tfrac{1}{2}\Delta R_{il}\left( h^2a_l + H_a\int_0^h W_t^a\diff{t} \right) \right]
\end{align}

\section{Uncertainty Propagation}
Suppose for any $i < k \leq j$, $(\Delta R_{ik}, \Delta v_{ik}, \Delta p_{ik}) \sim \mathcal{MG}(U_k,S_k,V_k,\mu_k,\Sigma_k,P_k)$, we want to match the distribution of $(\Delta R_{ik+1}, \Delta v_{ik+1}, \Delta p_{ik+1})$ to a new MFG.
The parameters $\mu_k\in\real{6}$, $P_k\in\real{6\times 3}$, and $\Sigma_k\in\real{6\times 6}$ are factorized into the velocity and position components, as follows:
\begin{gather}
	\mu_k = \begin{bmatrix} \mu_{v_k}^T & \mu_{p_k}^T \end{bmatrix}^T, \qquad
	P_k = \begin{bmatrix} P_{v_k} \\ P_{p_k} \end{bmatrix}, \qquad 
	\Sigma_k = \begin{bmatrix} \Sigma_{vv_k} & \Sigma_{vp_k} \\ \Sigma_{pv_k} & \Sigma_{pp_k} \end{bmatrix}.
\end{gather}

\subsection{Moments for $\Delta R_{ik+1}$}
Decompose $\Delta R_{ik+1}$ as
\begin{align*}
	\Delta R_{ik+1} = \Delta R_{ik} \exp(h\hat{\Omega}_k + (H_g\Delta W_t^g)^\wedge) = \Delta R_{ik} \exp((H_g\Delta t^g)^\wedge + C(h,\Delta W_t^g)) \exp(h\hat{\Omega}_k),
\end{align*}
where $C(h,\Delta W_t^g) = -\frac{1}{2}[h\hat{\Omega}_k+(H_g\Delta W_t^g)^\wedge, h\hat{\Omega}_k] + \cdots \sim o(h)$.
First
\color{blue}
\begin{align}
	\expect{\Delta R_{ik+1}} &= \expect{\Delta R_{ik}} \expect{\exp\left( (H_g\Delta W_t^g)^\wedge + C(h,\Delta W_t^g) \right)} \exp(h\hat{\Omega}_k) \nonumber \\
	&= \expect{\Delta R_{ik}} \expect{ \sum_{n=1}^\infty \tfrac{1}{n!} \left( (H_g\Delta W_t^g)^\wedge + C(h,\Delta W_t^g)^n \right) } \exp(h\hat{\Omega}_k) \nonumber \\
	&= \expect{\Delta R_{ik}} \left( I_{3\times 3} + \tfrac{h}{2}(G_g-\tr{G_g}I_{3\times 3}) \right) \exp(h\hat{\Omega}_k) + O(h^2),
\end{align}
\color{black}
where $G_g = H_gH_g^T$.
Then $U_{k+1}$, $V_{k+1}$ are given by $U_{k+1}D_{k+1}V_{k+1}^T = \expect{\Delta R_{ik+1}}$, and $S_{k+1}$ can be solved from $D_{k+1}$.
Let $\nu_{\Delta R_{ik+1}} = (Q_{k+1}S_{k+1} - S_{k+1}Q_{k+1}^T)^\vee$, where $Q_{k+1} = U_{k+1}^T \Delta R_{ik+1} V_{k+1}$, then
\color{blue}
\begin{align}
	\expect{\nu_{\Delta R_{ik+1}}} = 0.
\end{align}
\color{black}

For any $A\in\real{3\times 3}$, let $\Lambda(A) = (U_{k+1}^TAV_{k+1}S_{k+1} - S_{k+1}V_{k+1}^TA^TU_{k+1})^\vee$.
Also, denote $\delta R_k = \exp(h\hat{\Omega}_k)$, and $\xi = (H_g\Delta W_t^g)^\wedge + C(h,\Delta W_t^g)$, then
\color{blue}
\begin{align} \label{eqn:EvR(k+1)vR(k+1)}
	\expect{\nu_{\Delta R_{ik+1}} \nu_{\Delta R_{ik+1}}^T} &= \sum_{m,n=1}^{\infty} \tfrac{1}{m!n!} \expect{\Lambda\left( \Delta R_{ik} \xi^m \delta R_k \right) \Lambda\left( \Delta R_{ik} \xi^n \delta R_k \right)^T} \nonumber \\
	&= \expect{\Lambda\left( \Delta R_{ik} \delta R_k \right) \Lambda\left( \Delta R_{ik} \delta R_k \right)^T} + \expect{\Lambda\left( \Delta R_{ik} (H_g\Delta W_t^g)^\wedge \delta R_k \right) \Lambda\left( \Delta R_{ik} (H_g\Delta W_t^g)^\wedge \delta R_k \right)^T} \nonumber \\
	&\quad + \tfrac{1}{2}\expect{\Lambda\left( \Delta R_{ik} \delta R_k \right) \Lambda\left( \Delta R_{ik} ((H_g\Delta W_t^g)^\wedge)^2 \delta R_k \right)^T} \nonumber \\
	&\quad + \tfrac{1}{2}\expect{\Lambda\left( \Delta R_{ik} ((H_g\Delta W_t^g)^\wedge)^2 \delta R_k \right) \Lambda\left( \Delta R_{ik} \delta R_k \right)^T} + O(h^2).
\end{align}
\color{black}
Let $\tilde{U} = U_{k+1}^TU_k$, $\tilde{V} = V_{k+1}^T\exp(-h\hat{\Omega}_k)V_k$, and $\tilde{S} = \tilde{U}^TS_{k+1}\tilde{V}$.
Also, let $Q_k = U_k^T\Delta R_{ik}V_k$, and $\tilde{\nu}_{\Delta R_{ik}} = (Q_k\tilde{S}^T - \tilde{S}Q_k^T)^\vee$, then we have
\begin{align*}
	\Lambda(\Delta R_{ik} \delta R_k) &= (U_{k+1}^TU_kQ_kV_k^T\delta R_kV_{k+1}S_{k+1} - S_{k+1}V_{k+1}^T\delta R_k^TV_kQ_k^TU_k^TU_{k+1})^\vee \\
	&= (\tilde{U}Q_k\tilde{V}^TS_{k+1} - S_{k+1}\tilde{V}Q_k^T\tilde{U}^T)^\vee \\
	&= \tilde{U} (Q_k\tilde{V}^TS_{k+1}\tilde{U} - \tilde{U}^TS_{k+1}\tilde{V}Q_k^T)^\vee \\
	&= \tilde{U} (Q_k\tilde{S}^T - \tilde{S}Q_k^T)^\vee \\
	&= \tilde{U} \tilde{\nu}_{\Delta R_{ik}}.
\end{align*}
Thus the first term on the right hand side of \eqref{eqn:EvR(k+1)vR(k+1)} can be written as
\begin{align}
	\expect{\Lambda\left( \Delta R_{ik} \delta R_k \right) \Lambda\left( \Delta R_{ik} \delta R_k \right)^T} = \tilde{U} \expect{\tilde{\nu}_{\Delta R_{ik}} \tilde{\nu}_{\Delta R_{ik}}^T} \tilde{U}^T.
\end{align}
Next, let $\Gamma_k = \left( \tr{Q_k\tilde{S}^T}I_{3\times 3} - Q_k\tilde{S}^T \right) Q_k$, then we have
\begin{align*}
	\Lambda\left( \Delta R_{ik} (H_g\Delta W_t^g)^\wedge \delta R_k \right) &= (U_{k+1}^T \Delta R_{ik} (H_g\Delta W_t^g)^\wedge \delta R_kV_{k+1}S_{k+1} + S_{k+1}V_{k+1}^T\delta R_k^T (H_g\Delta W_t^g)^\wedge \Delta R_{ik}^TU_{k+1})^\vee \\
	&= (\tilde{U}Q_kV_k^T(H_g\Delta W_t^g)^\wedge \delta R_kV_{k+1}S_{k+1} + S_{k+1}V_{k+1}^T\delta R_k^T (H_g\Delta W_t^g)^\wedge V_kQ_k^T\tilde{U}^T)^\vee \\
	&= (\tilde{U}Q_k(V_k^TH_g\Delta W_t^g)^\wedge \tilde{V}^TS_{k+1} + S_{k+1}\tilde{V}(V_k^TH_g\Delta W_t^g)^\wedge Q_k^T\tilde{U}^T)^\vee \\
	&= \tilde{U}(Q_k(V_k^TH_g\Delta W_t^g)^\wedge \tilde{S}^T + \tilde{S}(V_k^TH_g\Delta W_t^g)^\wedge Q_k^T)^\vee \\
	&= \tilde{U}((Q_kV_k^TH_g\Delta W_t^g)^\wedge Q_k\tilde{S}^T + \tilde{S}Q_k^T(Q_kV_k^TH_g\Delta W_t^g)^\wedge)^\vee \\
	&= \tilde{U}\left( \tr{Q_k\tilde{S}^T}I_{3\times 3} - Q_k\tilde{S}^T \right) Q_kV_k^TH_g\Delta W_t^g \\
	&= \tilde{U}\Gamma_kV_k^TH_g\Delta W_t^g.
\end{align*}
Thus, the second term on the right hand side of \eqref{eqn:EvR(k+1)vR(k+1)} can be written as
\begin{align}
	\expect{\Lambda\left( \Delta R_{ik} (H_g\Delta W_t^g)^\wedge \delta R_k \right) \Lambda\left( \Delta R_{ik} (H_g\Delta W_t^g)^\wedge \delta R_k \right)^T} &= \tilde{U} \expect{\Gamma_kV_k^TH_g\Delta W_t^g (H_g\Delta W_t^g)^TV_k\Gamma_k^T} \tilde{U}^T \nonumber \\
	&= h\tilde{U} \expect{\Gamma_kV_k^TG_gV_k\Gamma_k^T} \tilde{U}^T.
\end{align}
Lastly, let $\tilde{\tilde{V}} = V_{k+1}^T\exp(-h\hat{\Omega}_k)G_g^TV_k$, $\tilde{\tilde{S}} = \tilde{U}^TS_{k+1}\tilde{\tilde{V}}$ and $\tilde{\tilde{\nu}}_{\Delta R_{ik}} = (Q_k\tilde{\tilde{S}}^T - \tilde{\tilde{S}}Q_k^T)^\vee$, then we have
\begin{align*}
	\Lambda\left( \Delta R_{ik}G_u\delta R_k \right) &= (U_{k+1}^T \Delta R_{ik} G_g \delta R_kV_{k+1}S_{k+1} + S_{k+1}V_{k+1}^T\delta R_k^T G_g^T \Delta R_{ik}^TU_{k+1})^\vee \\
	&= (\tilde{U}Q_k\tilde{\tilde{V}}^TS_{k+1} - S_{k+1}\tilde{\tilde{V}}Q_k^T\tilde{U}^T)^\vee \\
	&= \tilde{U}(Q_k\tilde{\tilde{S}}^T - \tilde{\tilde{S}}Q_k^T)^\vee \\
	&= \tilde{U}\tilde{\tilde{\nu}}_{\Delta R_{ik}}.
\end{align*}
Thus, the third term on the right hand side of \eqref{eqn:EvR(k+1)vR(k+1)} can be written as
\begin{align}
	&\expect{\Lambda\left( \Delta R_{ik} \delta R_k \right) \Lambda\left( \Delta R_{ik} ((H_g\Delta W_t^g)^\wedge)^2 \delta R_k \right)^T} = h\tilde{U} \expect{\tilde{\nu}_{\Delta R_{ik}} \Lambda\left( \Delta R_{ik} (G_g-\tr{G_g}I_{3\times 3}) \delta R_k \right)^T} \nonumber \\
	&\qquad \qquad = h\tilde{U} \expect{\tilde{\nu}_{\Delta R_{ik}} \Lambda\left( \Delta R_{ik}G_g\delta R_k \right)^T} - h\tr{G_g}\tilde{U} \expect{\tilde{\nu}_{\Delta R_{ik}} \Lambda\left( \Delta R_{ik}\delta R_k \right)^T} \nonumber \\
	&\qquad \qquad = h\tilde{U}\expect{ \tilde{\nu}_{\Delta R_{ik}} \tilde{\tilde{\nu}}_{\Delta R_{ik}}^T }\tilde{U}^T - h\tr{G_g}\tilde{U} \expect{\tilde{\nu}_{\Delta R_{ik}} \tilde{\nu}_{\Delta R_{ik}}^T} \tilde{U}^T.
\end{align}

\subsection{Moments for $\Delta v_{ik+1}$ and $\Delta p_{ik+1}$}
Next, we consider moments involved with $\Delta v_{ik+1}$ and $\Delta p_{ik+1}$.
First, we have
\color{blue}
\begin{gather}
	\expect{\Delta v_{ik+1}} = \expect{\Delta v_{ik}} + \expect{\Delta R_{ik}(ha_k + H_a\Delta W_t^a)} = \expect{\Delta v_{ik}} + h\expect{\Delta R_{ik}}a_k, \\
	\expect{\Delta p_{ik+1}} = \expect{\Delta p_{ik}} + h\expect{\Delta v_{ik}} + \tfrac{1}{2}h^2\expect{\Delta R_{ik}}a_k.
\end{gather}
\color{black}

Let $G_a = H_aH_a^T$, then for the second order moments, we have
\color{blue}
\begin{align}
	\expect{\Delta v_{ik+1} \Delta v_{ik+1}^T} &= \expect{\Delta v_{ik} \Delta v_{ik}^T} + \expect{\Delta R_{ik}(ha_k+H_a\Delta W_t^a)(ha_k+H_a\Delta W_t^a)^T\Delta R_{ik}^T} \nonumber \\
	&\qquad + h\expect{\Delta v_{ik}a_k^T\Delta R_{ik}^T} + h\expect{\Delta R_{ik}a_k\Delta v_{ik}^T},
\end{align}
\color{black}
where the second term on the right hand side can be written as
\begin{align}
	\expect{\Delta R_{ik}(ha_k+H_a\Delta W_t^a)(ha_k+H_a\Delta W_t^a)^T\Delta R_{ik}^T} = \expect{\Delta R_{ik} (h^2a_ka_k^T + hG_a)\Delta R_{ik}^T},
\end{align}
and the third term on the right hand side can be written as
\begin{align}
	\expect{\Delta v_{ik}a_k^T\Delta R_{ik}^T} = \mu_{v_k}a_k^T\expect{\Delta R_{ik}^T} + P_{v_k}\expect{\nu_{\Delta R_{ik}} a_k^T \Delta R_{ik}^T}.
\end{align}
Similarly, we have
\color{blue}
\begin{align}
	\expect{\Delta p_{ik+1} \Delta p_{ik+1}^T} &= \expect{\Delta p_{ik} \Delta p_{ik}^T} + h\expect{\Delta p_{ik} \Delta v_{ik}^T} + h\expect{\Delta v_{ik} \Delta p_{ik}^T} + h^2\expect{\Delta v_{ik} \Delta v_{ik}} \nonumber \\
	&\quad + \tfrac{1}{2}h^2\expect{\Delta p_{ik}a_k^T\Delta R_{ik}^T} + \tfrac{1}{2}h^2\expect{\Delta R_{ik}a_k\Delta p_{ik}^T} + \tfrac{1}{2}h^3\expect{\Delta v_{ik}a_k^T\Delta R_{ik}^T} \nonumber \\
	&\quad + \tfrac{1}{2}h^3\expect{\Delta R_{ik}a_k\Delta v_{ik}^T} + \tfrac{1}{4}\expect{\Delta R_{ik} \left(h^2a_k + H_a\int_0^h W_t^a\diff{t}\right) \left(h^2a_k + H_a\int_0^h W_t^a\diff{t}\right)^T \Delta R_{ik}^T}.
\end{align}
\color{black}
The fifth term on the right hand side can be written as
\begin{align}
	\expect{\Delta p_{ik}a_k^T\Delta R_{ik}^T} = \mu_{p_k}a_k^T\expect{\Delta R_{ik}^T} + P_{p_k}\expect{\nu_{\Delta R_{ik}}a_k^T\Delta R_{ik}^T}.
\end{align}
Note that $\int_0^h W_t^a\diff{t} \sim \mathcal{N}(0,\tfrac{1}{3}h^3I_{3\times 3})$, so the last term on the right hand side can be written as
\begin{align}
	\expect{\Delta R_{ik} \left(h^2a_k + H_a\int_0^h W_t^a\diff{t}\right) \left(h^2a_k + H_a\int_0^h W_t^a\diff{t}\right)^T \Delta R_{ik}^T} = \expect{\Delta R_{ik} \left(h^4a_ka_k^T + \tfrac{1}{3}h^3G_a\right) \Delta R_{ik}}.
\end{align}
Lastly, we have
\color{blue}
\begin{align} \label{eqn:Ev(k+1)p(k+1)}
	\expect{\Delta v_{ik+1} \Delta p_{ik+1}^T} &= \expect{\Delta v_{ik} \Delta p_{ik}^T} + h\expect{\Delta v_{ik} \Delta v_{ik}^T} + h\expect{\Delta R_{ik}a_k\Delta P_{ik}^T} + \tfrac{1}{2}h^2\expect{\Delta v_{ik}a_k^T\Delta R_{ik}^T} \nonumber \\
	&\quad + h^2\expect{\Delta R_{ik}a_k\Delta v_{ik}^T} + \tfrac{1}{2} \expect{\Delta R_{ik}(ha_k + H_a\Delta W_t^a) \left(h^2a_k + H_a\int_0^t W_t^a\diff{t}\right)^T \Delta R_{ik}^T}
\end{align}
\color{black}
By Ito's isometry, we have
\begin{align*}
	\expect{\int_0^h \diff{W_t^a} \left( \int_0^h W_t^a \diff{t} \right)^T} = \expect{\int_0^h \diff{W_t^a} \left( \int_0^h (h-t) \diff{W_t^a} \right)^T} 
	=\expect{\int_0^h (h-t) \diff{t}} I_{3\times 3} = \tfrac{1}{2}h^2I_{3\times 3}.
\end{align*}
Thus the last term on the right hand side of \eqref{eqn:Ev(k+1)p(k+1)} can be written as
\begin{align}
	\expect{\Delta R_{ik}(ha_k + H_a\Delta W_t^a) \left(h^2a_k + H_a\int_0^t W_t^a\diff{t}\right)^T \Delta R_{ik}^T} = \expect{\Delta R_{ik} \left(h^3a_ka_k^T + \tfrac{1}{2}h^2G_a\right) \Delta R_{ik}^T}.
\end{align}

Next, we consider the correlation between $\Delta v_{ik+1}$, $\Delta p_{ik+1}$, and $\nu_{\Delta R_{ik+1}}$.
First, we have
\color{blue}
\begin{align}
	\expect{\Delta v_{ik+1} \nu_{\Delta R_{ik+1}}^T} = \expect{\Delta v_{ik} \nu_{\Delta R_{ik+1}}^T} + h\expect{\Delta R_{ik}a_k\nu_{\Delta R_{ik+1}}^T}.
\end{align}
\color{black}
The first term on the right hand side can be calculated as
\begin{align}
	\expect{\Delta v_{ik} \nu_{\Delta R_{ik+1}}^T} &= \mu_{v_k} \expect{\nu_{\Delta R_{ik+1}}^T} + P_{v_k} \expect{\nu_{\Delta R_{ik}} \nu_{\Delta R_{ik+1}}^T} \nonumber \\
	&= P_{v_k}\sum_{n=1}^\infty \tfrac{1}{n!} \expect{\nu_{\Delta R_{ik}} \Lambda\left( \Delta R_{ik}\xi^n\delta R_k \right)^T} \nonumber \\
	&= P_{v_k}\expect{\nu_{\Delta R_{ik}} \Lambda\left( \Delta R_{ik}\delta R_k \right)^T} + \tfrac{1}{2} P_{v_k} \expect{\nu_{\Delta R_{ik}} \Lambda\left( \Delta R_{ik} ((H_g\Delta W_t^g)^\wedge)^2 \delta R_k \right)^T} + O(h^2) \nonumber \\
	&= P_{v_k}\expect{\nu_{\Delta R_{ik}} \tilde{\nu}_{\Delta R_{ik}}^T}\tilde{U}^T + \tfrac{1}{2} hP_{v_k}\expect{\nu_{\Delta R_{ik}} \tilde{\tilde{\nu}}_{\Delta R_{ik}}^T} \tilde{U}^T - \tfrac{1}{2}h\tr{G_g} P_{v_k}\expect{\nu_{\Delta R_{ik}} \tilde{\nu}_{\Delta R_{ik}}^T} \tilde{U}^T + O(h^2) \nonumber \\
	&= \tfrac{1}{2} hP_{v_k}\expect{\nu_{\Delta R_{ik}} \tilde{\tilde{\nu}}_{\Delta R_{ik}}^T} \tilde{U}^T + \left( 1- \tfrac{1}{2}h\tr{G_g} \right) P_{v_k}\expect{\nu_{\Delta R_{ik}} \tilde{\nu}_{\Delta R_{ik}}^T} \tilde{U}^T + O(h^2).
\end{align}
Similarly, the second term on the right hand side can be calculated as
\begin{align}
	\expect{\Delta R_{ik}a_k\nu_{\Delta R_{ik+1}}^T} &= \expect{\Delta R_{ik}a_k\Lambda(\Delta R_{ik}\delta R_k)^T} + \tfrac{1}{2}h\expect{\Delta R_{ik}a_k\Lambda\left( \Delta R_{ik} ((H_g\Delta W_t^g)^\wedge)^2 \delta R_k \right)^T} + O(h^2) \nonumber \\
	&= \tfrac{1}{2}h U_k\expect{Q_kV_k^Ta_k\tilde{\tilde{\nu}}_{\Delta R_{ik}}^T} \tilde{U}^T + \left( 1- \tfrac{1}{2}h\tr{G_g} \right) U_k\expect{Q_kV_k^Ta_k\tilde{\nu}_{\Delta R_{ik}}^T} \tilde{U}^T + O(h^2).
\end{align}
Finally,
\color{blue}
\begin{align}
	\expect{\Delta p_{ik+1} \nu_{\Delta R_{ik+1}}^T} = \expect{\Delta p_{ik} \nu_{\Delta R_{ik+1}}^T} + h\expect{\Delta v_{ik}\nu_{\Delta R_{ik+1}}^T} + \tfrac{1}{2}h^2\expect{\Delta R_{ik}a_k\nu_{\Delta R_{ik+1}}^T},
\end{align}
\color{black}
where the first term on the right hand side can be written as
\begin{align}
	\expect{\Delta p_{ik} \nu_{\Delta R_{ik+1}}^T} = \tfrac{1}{2}hP_{p_k}\expect{\nu_{\Delta R_{ik}} \tilde{\tilde{\nu}}_{\Delta R_{ik}}^T} \tilde{U}^T + \left( 1- \tfrac{1}{2}h\tr{G_g} \right) P_{p_k}\expect{\nu_{\Delta R_{ik}} \tilde{\nu}_{\Delta R_{ik}}^T} \tilde{U}^T + O(h^2).
\end{align}

\subsection{Gradient}

Having the density function for $(\Delta R_{ij}, \Delta v_{ij}, \Delta p_{ij})$, we want to calculate its gradient with respect to $(\delta \phi_i, \delta \phi_j, \delta v_i, \delta v_j, \delta p_i, \delta p_j)$.
First, $\Delta R_{ij}, \Delta v_{ij}, \Delta p_{ij}$ can be written as
\begin{align}
	\Delta R_{ij} &= [R_i\exp(\delta \hat{\phi}_i)]^T R_j\exp(\delta \hat{\phi}_j), \\
	\Delta v_{ij} &= [R_i\exp(\delta \hat{\phi}_i)]^T (v_j+\delta v_j - (v_i+\delta v_i) - \Delta t_{ij}g), \\
	\Delta p_{ij} &= [R_i\exp(\delta \hat{\phi}_i)]^T (p_j+\delta p_j - (p_i+\delta p_i) - \Delta t_{ij}(v_i+\delta v_i) - \tfrac{1}{2}\Delta t_{ij}^2g).
\end{align}
The negative log density function is
\begin{align}
	f = -\tr{QS} + \tfrac{1}{2}(\Delta x_{ij} - \mu_c)^T \Sigma_c^{-1} (\Delta x_{ij} - \mu_c),
\end{align}
where $x_{ij} = \begin{bmatrix} \Delta v_{ij}^T & \Delta p_{ij}^T \end{bmatrix}^T$.

Then we have
\color{blue}
\begin{align}
	\frac{\partial f}{\partial \delta \phi_i} = -\frac{\partial \tr{QS}}{\partial \delta \phi_i} + \frac{\partial (\Delta x_{ij} - \mu_c)^T}{\partial \delta \phi_i} \Sigma_c^{-1} (\Delta x_{ij} - \mu_c).
\end{align}
\color{black}
The first term on the right hand side can be calculated as
\begin{align}
	\frac{\partial \tr{QS}}{\partial \delta \phi_i} &\approx \frac{\partial \tr{U^T[R_i(I_{3\times 3} + \delta \hat{\phi}_i)]^TR_jVS}}{\partial \phi_i}
	= -\frac{\partial \tr{\delta\hat{\phi}_i R_i^TR_jF^T}}{\partial \delta\phi_i} \nonumber \\
	&= \frac{\partial \delta\phi_i^T (R_i^TR_jF^T - FR_j^TR_i)^\vee}{\partial \phi_i} = (R_i^TR_jF^T - FR_j^TR_i)^\vee.
\end{align}
The second term can be calculated as
\begin{align}
	\frac{\partial \Delta v_{ij}}{\partial \delta\phi_i} &\approx \frac{\partial [R_i(I_{3\times 3} + \hat{\phi}_i)]^T (v_j-v_i-\Delta t_{ij}g)}{\partial \delta\phi_i} = \left[ R_i^T(v_j-v_i-\Delta t_{ij}g) \right]^\wedge, \\
	\frac{\partial \Delta p_{ij}}{\partial \delta\phi_i} &\approx \frac{\partial [R_i(I_{3\times 3} + \hat{\phi}_i)]^T (p_j-p_i-\Delta t_{ij}v_i - \tfrac{1}{2}\Delta t_{ij}^2g)}{\partial \delta\phi_i} = \left[ R_i^T(p_j-p_i-\Delta t_{ij}v_i - \tfrac{1}{2}\Delta t_{ij}^2g) \right]^\wedge.
\end{align}
And
\begin{align}
	\frac{\partial \mu_c}{\partial \delta\phi_i} &= P \frac{\partial (QS-SQ^T)^\vee}{\partial \delta\phi_i} \approx P \frac{\partial \left(U^T[R_i(I_{3\times 3}+\delta\hat{\phi}_i)]^TR_jVS - SV^TR_j^TR_i(I_{3\times 3}+\delta\hat{\phi}_i)U\right)^\vee}{\partial \delta\phi_i} \nonumber \\
	&= -PU^T \frac{\partial \left(\delta\hat{\phi}R_i^TR_jF^T + FR_j^TR_i\delta\hat{\phi}\right)^\vee}{\partial \delta\phi_i} = -PU^T \frac{\partial \left(\tr{R_i^TR_jF^T}I_{3\times 3} - R_i^TR_jF^T\right) \delta\phi_i}{\partial \delta\phi_i} \nonumber \\
	&= PU^T\left(R_i^TR_jF^T - \tr{R_i^TR_jF^T}I_{3\times 3}\right).
\end{align}
Next, the gradient with respect to $\delta\phi_j$ is
\color{blue}
\begin{align}
	\frac{\partial f}{\partial \delta\phi_j} = -\frac{\partial \tr{QS}}{\partial \delta \phi_j} - \frac{\partial \mu_c^T}{\partial \delta\phi_j} \Sigma_c^{-1} (\Delta x_{ij} - \mu_c).
\end{align}
\color{black}
Similarly, the first term on the right hand side can be calculated as
\begin{align}
	\frac{\partial \tr{QS}}{\partial \delta\phi_j} \approx \frac{\partial \tr{U^TR_i^TR_j(I_{3\times 3}+\hat{\phi}_j)VS}}{\partial \delta\phi_j} = \frac{\partial \tr{\hat{\phi}_jF^TR_i^TR_j}}{\partial \delta\phi_j} = -(F^TR_i^TR_j - R_j^TR_iF)^\vee.
\end{align}
And the second term on the right hand side can be calculated as
\begin{align}
	\frac{\partial \mu_c}{\partial \delta\phi_j} &\approx P\frac{\partial \left(U^TR_i^TR_j(I_{3\times 3}+\hat{\phi}_j)VS - SV^T[R_j(I_{3\times 3}+\hat{\phi}_j)]^TR_iU\right)^\vee}{\partial \delta\phi_j} \nonumber \\
	&= PU^TR_i^TR_j \frac{\partial \left(\hat{\phi}_jF^TR_i^TR_j + R_j^TR_iF\hat{\phi}_j\right)^\vee}{\partial \delta\phi_j} = PU^TR_i^TR_j \left(\tr{F^TR_i^TR_j}I_{3\times 3} - F^TR_i^TR_j\right).
\end{align}

In addition, the gradient with respect to $\delta v_i$ and $\delta v_j$ are
\color{blue}
\begin{align}
	\frac{\partial f}{\partial \delta v_i} = \frac{\partial \Delta x_{ij}^T}{\partial \delta v_i} \Sigma_c^{-1} (\Delta x_{ij}-\mu_c), \qquad \frac{\partial f}{\partial \delta v_j} = \frac{\partial \Delta x_{ij}^T}{\partial \delta v_j} \Sigma_c^{-1} (\Delta x_{ij}-\mu_c).
\end{align}
\color{black}
And for $\delta v_i$, we have
\begin{align}
	\frac{\partial \Delta v_{ij}}{\partial \delta v_i} = -R_i^T, \qquad \frac{\partial \Delta p_{ij}}{\partial \delta v_i} = -\Delta t_{ij} R_i^T.
\end{align}
Similarly, for $\delta v_j$, we have
\begin{align}
	\frac{\partial \Delta v_{ij}}{\partial \delta v_j} = R_i^T, \qquad \frac{\partial \Delta p_{ij}}{\partial \delta v_j} = 0.
\end{align}

Finally, the gradient with respect to $\delta p_i$ and $\delta p_j$ are
\color{blue}
\begin{align}
	\frac{\partial f}{\partial \delta p_i} = \frac{\partial \Delta x_{ij}^T}{\partial \delta p_i} \Sigma_c^{-1} (\Delta x_{ij}-\mu_c), \qquad \frac{\partial f}{\partial \delta p_j} = \frac{\partial \Delta x_{ij}^T}{\partial \delta p_j} \Sigma_c^{-1} (\Delta x_{ij}-\mu_c).
\end{align}
\color{black}
And for $\delta p_i$, we have
\begin{align}
	\frac{\partial \Delta v_{ij}}{\partial \delta p_i} = 0, \qquad \frac{\partial \Delta p_{ij}}{\partial \delta p_i} = -R_i^T.
\end{align}
Similarly, for $\delta p_j$, we have
\begin{align}
	\frac{\partial \Delta v_{ij}}{\partial \delta p_j} = 0, \qquad \frac{\partial \Delta p_{ij}}{\partial \delta p_j} = R_i^T.
\end{align}

\subsection{Pre-integration}

From the highlighted formulae, we can recover match $(\Delta R_{ik+1}, \Delta v_{ik+1}, \Delta p_{ik+1})$ to a new MFG, by assuming $(\Delta R_{ik}, \Delta v_{ik}, \Delta p_{ik})$ follows an MFG.
To find the MFG for $(\Delta R_{ij}, \Delta v_{ij}, \Delta p_{ij})$, we start from $(\Delta R_{ii+1}, \Delta v_{ii+1}, \Delta p_{ii+1})$:
\begin{align}
	\Delta R_{ii+1} &= \exp\left( h\hat{\Omega}_i + (H_g\Delta W_t^g)^\wedge \right), \\
	\Delta v_{ii+1} &= ha_i + H_a\Delta W_t^a, \\
	\Delta p_{ii+1} &= \tfrac{1}{2} \left( h^2a_i + H_a\int_0^h W_t^a\diff{t}\right).
\end{align}
So $(\Delta R_{ii+1}, \Delta v_{ii+1}, \Delta p_{ii+1})$ are independent, and
\begin{gather}
	\expect{\Delta R_{ii+1}} = \left( I_{3\times 3} + \tfrac{h}{2}(G_g-\tr{G_g}I_{3\times 3}) \right) \exp(h\hat{\Omega}_k) + O(h^2), \\
	\Delta v_{ii+1} \sim \mathcal{N}(ha_i, hG_a), \\
	\Delta p_{ii+1} \sim \mathcal{N}(\tfrac{1}{2}h^2a_i, \tfrac{1}{3}h^3G_a).
\end{gather}
Let $x = (R,v,p)$, and $\Delta x_{ij} = (\Delta R_{ij}, \Delta v_{ij}, \Delta p_{ij})$, then $\Delta x_{ij} (x_i,x_j)$ is a function of $x_i$ and $x_j$ according to \eqref{eqn:Delta R}, \eqref{eqn:Delta v} and \eqref{eqn: Delta p}.
The propagated MFG at time $t_j$ gives the density for $p(\Delta x_{ij} | \mathcal{I}_{ij})$, where $\mathcal{I}_{ij}$ denotes all IMU measurements at time $t_k$, $k=i,\ldots,j-1$.
Suppose the prior $p(\Delta x_{ij})$ is uniform, then
\begin{align}
	p(\mathcal{I}_{ij} | \Delta x_{ij}) = \frac{p(\Delta x_{ij} | \mathcal{I}_{ij}) p(\mathcal{I}_{ij})}{p(\Delta x_{ij})} \propto p(\Delta x_{ij}|\mathcal{I}_{ij}).
\end{align}

From another point of view, using the notations in \cite{forster2016manifold}, let $\mathcal{K}_k = \{i_0,\ldots,i_n\}$ be all keyframes, $\mathcal{I}_k = \{\mathcal{I}_{i_j i_{j+1}}\}_{i_j\in\mathcal{K}_k}$, $\mathcal{C}_k = \{\mathcal{C}_i\}_{i\in\mathcal{K}}$, and $\mathcal{X}_k = \{x_i\}_{i\in\mathcal{K}_k}$.
Then the posterior density can be factorized into
\begin{align}
	p(\mathcal{X}_k | \mathcal{C}_k, \mathcal{I}_k) &= \frac{p(\mathcal{C}_k | \mathcal{X}_k, \mathcal{I}_k) p(\mathcal{X}_k | \mathcal{I}_k)}{p(\mathcal{C}_k | \mathcal{I}_k)} \propto p(\mathcal{C}_k | \mathcal{X}_k) p(\mathcal{X}_k | \mathcal{I}_k).
\end{align}
The density $p(\mathcal{X}_k | \mathcal{I}_k)$ can further be decomposed into
\begin{align}
	p(\mathcal{X}_k | \mathcal{I}_k) &= p(x_{i_0}) p(x_{i_1},\ldots,x_{i_n} | \mathcal{I}_k, x_{i_0}) \nonumber \\
	&= p(x_{i_0}) p(x_{i_2},\ldots,x_{i_n} | \mathcal{I}_k, x_{i_0}, x_{i_1}) p(x_{i_1} | \mathcal{I}_k, x_{i_0}) \nonumber \\
	&= p(x_{i_0}) p(x_{i_2},\ldots,x_{i_n} | \mathcal{I}_{i_1i_2}, \ldots, \mathcal{I}_{i_{n-1}i_n}, x_{i_1}) p(x_{i_1} | \mathcal{I}_{0i_1}, x_{i_0}) \nonumber \\
	&= p(x_{i_0}) \prod_{j=0}^{n-1} p(x_{i_j} | \mathcal{I}_{i_{j-1}i_j}, x_{i_{j-1}}) \nonumber \\
	&= p(x_{i_0}) \prod_{j=0}^{n-1} p(\Delta x_{i_{j-1}i_j} | \mathcal{I}_{i_{j-1}i_j})
\end{align}

\bibliographystyle{IEEEtran}
\bibliography{ref}

\end{document}

