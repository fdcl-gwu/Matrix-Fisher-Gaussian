\documentclass[10pt]{article}

\usepackage[letterpaper,margin=0.5in]{geometry}

\usepackage{amssymb,amsmath,amsthm}
\usepackage{bm}
\usepackage{graphicx,subcaption}

\newtheorem{definition}{Definition}
\newtheorem{proposition}{Proposition}
\newtheorem{theorem}{Theorem}
\newtheorem{lemma}{Lemma}
\newtheorem{remark}{Remark}

\title{\vspace{-4ex}\textbf{Position Estimation Using MFG\vspace{-4ex}}}
\date{}

\newcommand{\norm}[1]{\ensuremath{\left\| #1 \right\|}}
\newcommand{\fnorm}[1]{\ensuremath{\left\| #1 \right\|_\mathrm{F}}}
\newcommand{\tr}[1]{\ensuremath{\mathrm{tr}\left( #1 \right)}}
\newcommand{\etr}[1]{\ensuremath{\mathrm{etr}\left\{ #1 \right\}}}
\newcommand{\expect}[1]{\ensuremath{\mathrm{E}\left[ #1 \right]}}
\newcommand{\SO}{\ensuremath{\mathrm{SO(3)}}}
\newcommand{\real}[1]{\ensuremath{\mathbb{R}^{ #1 }}}
\newcommand{\diff}[1]{\ensuremath{\mathrm{d} #1}}
\newcommand{\expb}[1]{\ensuremath{\mathrm{exp}\left\{#1\right\}}}

\begin{document}

\maketitle

\section{Uncertainty Propagation}

Consider the following kinematics model:
\begin{align}
	R\diff{R} &= (\hat{\omega}+\hat{b}_g)\diff{t} + H_{gu}\diff{W_{gu}}, \\
	\diff{b_g} &= H_{gv}\diff{W_{gv}}, \\
	\diff{p} &= v\diff{t}, \\
	\diff{v} &= (R(a+ba)-g)\diff{t} + RH_{au}\diff{W_{au}}, \\
	\diff{b_a} &= H_{av}\diff{W_{av}},
\end{align}
where $(R,bg,p,v,ba)$ are attitude, gyroscope bias, position, velocity, and accelerometer bias of the IMU.
$\omega$ and $a$ are the measured angular velocity and linear velocity in IMU-frame, $(W_{gu},W_{gv},W_{au},W_{av})$ are velocity white noises and random walks for the gyroscope and accelerometer.
The discretized kinematics are
\begin{align}
	R(h) &= R(0)\exp\left\{ (\hat{\omega}(0) + \hat{b}_g(0))h + H_{gu}\Delta W_{gu} \right\} \\
	b_g(h) &= b_g(0) + H_{gv}\Delta W_{gv} \\
	p(h) &= p(0) + v(0)h \\
	v(h) &= v(0)-gh + R(0)\left( (a(0)+b_a(0))h + H_{au}\Delta W_{au} \right)\\
	b_a(h) &= b_a(0) + H_{av}\Delta W_{av}
\end{align}

Assume $(R,(b_g,p,v,b_a))$ follows a Matrix Fisher-Gaussian distribution at $t=0$ with parameters $(\mu,\Sigma,P,U,S,V)$, we wish to find the new distribution at $t=h$ with $h$ small.
We have calculated $\expect{R(h)}$, and therefore $U',S',V'$ at $t=h$, and also $\expect{\nu'_R}$, $\expect{b_g(h)\nu'^T_R}$, $\expect{\nu'_R\nu'^T_R}$ where $\nu'_R = (Q'S'-S'Q'^T)^\vee$ with $Q' = U'^TR(h)V'$.

\subsection{$\expect{x(h)}$}
It is clear that
\begin{align*}
	\expect{b_g(h)} &= \expect{b_g(0)} \\
	\expect{b_a(h)} &= \expect{b_a(0)} \\
	\expect{p(h)} &= \expect{p(0)}+h\expect{v(0)}.
\end{align*}
So the only non-trivial calculation is $\expect{v(h)}$.
Let
\begin{gather*}
	\mu = \begin{bmatrix}
		\mu_{b_g} \\ \mu_p \\ \mu_v \\ \mu_{b_a}
	\end{bmatrix}, \qquad
	P = \begin{bmatrix}
		P_{b_g} \\ P_p \\ P_v \\ P_{b_a}
	\end{bmatrix}, \\
	\Sigma = \begin{bmatrix}
		\Sigma_{b_gb_g} & \Sigma_{b_gp} & \Sigma_{b_gv} & \Sigma_{b_gb_a} \\
		\Sigma_{pb_g} & \Sigma_{pp} & \Sigma_{pv} & \Sigma_{pb_a} \\
		\Sigma_{vb_g} & \Sigma_{vp} & \Sigma_{vv} & \Sigma_{vb_a} \\
		\Sigma_{b_ab_g} & \Sigma_{b_ap} & \Sigma_{b_av} & \Sigma_{b_ab_a}
	\end{bmatrix}.
\end{gather*}

\begin{lemma}
	If $(R,x)\in\SO\times\real{3}$ follows $\mathcal{MG}(\mu,\Sigma,P,U,S,V)$, then $\expect{Rx} = \expect{R}\mu + \expect{RP\nu_R}$.
\end{lemma}
\begin{proof}
	This is immediate by integrating $x$ first.
\end{proof}

Then it is immediate that
\begin{equation}
	\expect{v(t)} = \expect{v(0)} + h\expect{R(0)}(a(0)+\mu_{b_a}) + h\expect{R(0)P_{b_a}\nu_R} - gh.
\end{equation}

\subsection{$\expect{x(h)x(h)^T}$}

Again only the terms involving $v(h)$ are non-trivial, in fact
\begin{align}
	\expect{b_g(h)b_g^T(h)} &= \expect{b_g(0)b_g^T(0)} + hG_{gv}, \\
	\expect{b_g(h)p^T(h)} &= \expect{b_g(0)p^T(0)} + h\expect{b_g(0)v^T(0)}, \\
	\expect{b_g(h)b_a^T(h)} &= \expect{b_g(0)b_a^T(0)}, \\
	\expect{p(h)p^T(h)} &= \expect{p(0)p^T(0)} + h\left(\expect{p(0)v^T(0)} + \expect{v(0)p^T(0)}\right) + h^2\expect{v(0)v^T(0)}, \\
	\expect{p(h)b_a^T(h)} &= \expect{p(0)b_a^T(0)} + h\expect{v(0)b_a^T(0)}, \\
	\expect{b_a(h)b_a^T(h)} &= \expect{b_a(0)b_a^T(0)} + hG_{av},
\end{align}
where $G_{gv} = H_{gv}H_{gv}^T$ and $G_{av} = H_{av}H_{av}^T$.
The terms involving $v(h)$ are calculated as follows:
\begin{align}
	\expect{v(h)b_g^T(h)} &= \expect{v(0)b_g^T(0)} + h\expect{R(0)a(0)b_g^T(0)} + h\expect{R(0)b_a(0)b_g^T(0)} - hg\expect{b_g(0)}^T,
\end{align}
where
\begin{align*}
	\expect{R(0)a(0)b_g^T(0)} &= \expect{R(0)}a(0)\mu_{b_g}^T + \expect{R(0)a(0)\nu^T_R}P^T_{b_g} \\
	\expect{R(0)b_a(0)b_g^T(0)} &= \expect{R(0)} \left( \Sigma_{b_ab_g} + \mu_{b_a}\mu_{b_g}^T - P_{b_a}(\tr{S}I-S)P_{b_g}^T \right) \\
	&\quad + \expect{R(0)\mu_{b_a}\nu^T_R}P^T_{b_g} + \expect{R(0)P_{b_a}\nu_R}\mu_{b_g}^T + \expect{R(0)P_{b_a}\nu_R\nu_R^T}P_{b_g}^T.
\end{align*}
Also,
\begin{align}
	\expect{v(h)p(h)^T} &= \expect{v(0)p^T(0)} + h\left( \expect{v(0)v^T(0)} + \expect{R(0)a(0)p^T(0)} + \expect{R(0)b_a(0)p^T(0)} \right) \nonumber \\
	&\quad + h^2\left( \expect{R(0)a(0)v^T(0)} + \expect{R(0)b_a(0)v^T(0)} \right) - hg\expect{p(0)+v(0)h}^T \\
	\expect{v(h)b_a^T(0)} &= \expect{v(0)b_a^T(0)} + h\expect{R(0)a(0)b_a^T(0)} + h\expect{R(0)b_a(0)b_a^T(0)} - hg\expect{b_a(0)}^T.
\end{align}
And the most difficult one,
\begin{align}
	\expect{v(h)v(h)^T} &= \expect{v(0)v(0)^T} + h\expect{R(0)(a(0)+b_a(0))v(0)^T} + h\expect{R(0)(a(0)+b_a(0))v(0)^T}^T \nonumber \\
	&\quad + h^2\expect{R(0)(a(0)+b_a(0))(a(0)+b_a(0))^TR(0)^T} + hG_{au}\expect{R(0)R(0)^T} + h^2gg^T \\
	&\quad - hg\expect{v(0)+hR(0)(a(0)+b_a(0))}^T - h\expect{v(0)+hR(0)(a(0)+b_a(0))}g^T,
\end{align}
where $G_{au} = H_{au}H_{au}^T$, and
\begin{align*}
	&\expect{R(0)b_a(0)b_a^T(0)R(0)^T} = \expect{R(0)\left( \Sigma_{b_ab_a} - P_{b_a}(\tr{S}I-S)P_{b_a}^T + \mu_{b_a}\mu_{b_a}^T \right)R(0)^T} \\
	&\qquad + \expect{R(0)\mu_{b_a}\nu_R^TP_{b_a}^TR(0)^T} + \expect{R(0)\mu_{b_a}\nu_R^TP_{b_a}^TR(0)^T}^T + \expect{R(0)P_{b_a}\nu_R\nu_R^TP_{b_a}^TR(0)^T}.
\end{align*}

\subsection{$\expect{x(h)\nu'^T_R}$}

We have already shown that
\begin{align}
	\expect{x(h)\nu'^T_R} = \expect{x(h)\nu'^T_{R(0)\delta R}} + h\expect{x(h)\nu'^T_{R(0)(\hat{b}_g(0)-\hat{\mu}_{b_g})\delta R}} + \frac{1}{2}\expect{x(h)\nu'^T_{R(0)((H_{gu}\Delta W_{gu})^\wedge)^2\delta R}} + O(h^2),
\end{align}
where $\nu'_A = (U'^TAV'S'-S'V'^TA^TU')^\vee$, and $\delta R = \exp((\hat{\omega}(0)+\hat{\mu}_{b_g})h)$.
If we only need to be accurate up to the first order, we may replace $x(h)$ by $x(0)$ in the last two terms in the above equation.
We have also shown that $\nu'_{R(0)\delta R} = \tilde{U}(Q\tilde{S}^T - \tilde{S}Q^T)^\vee$, where $\tilde{U} = U'^TU$, $\tilde{V} = V'^Te^{-h(\hat{\omega}(0)+\hat{\mu}_{b_g})}V$, and $\tilde{S} = \tilde{U}^TS'\tilde{V}$.
Then, the first term is calculated as follows
\begin{align}
	\expect{b_g(h)\nu'^T_{R(0)\delta R}} &= \expect{b_g(0)\nu'^T_{R(0)\delta R}}, \\
	\expect{p(h)\nu'^T_{R(0)\delta R}} &= \expect{p(0)\nu'^T_{R(0)\delta R}} + h\expect{v(0)\nu'^T_{R(0)\delta R}}, \\
	\expect{v(h)\nu'^T_{R(0)\delta R}} &= \expect{v(0)\nu'^T_{R(0)\delta R}} + h\expect{R(0)(a(0)+b_a(0))\nu'^T_{R(0)\delta R}} - hg\expect{\nu'^T_{R(0)\delta R}}, \\
	\expect{b_a(h)\nu'^T_{R(0)\delta R}} &= \expect{b_a(0)\nu'^T_{R(0)\delta R}}.
\end{align}

We give some example calculations:
\begin{equation}
	\expect{p(0)\nu'^T_{R(0)\delta R}} = \mu_p\expect{\nu'_{R(0)\delta R}}^T + P_p\expect{(QS-SQ^T)^\vee \left((Q\tilde{S}^T-\tilde{S}Q^T)^\vee\right)^T} \tilde{U}^T.
\end{equation}
\begin{align}
	\expect{R(0)b_a(0)\nu'^T_{R(0)\delta R}} &= U\expect{QV^T\mu_{b_a}\left((Q\tilde{S}^T-\tilde{S}Q^T)^\vee\right)^T}\tilde{U}^T \nonumber \\
	&\quad + U\expect{QV^TP_{b_a}(QS-SQ^T)^\vee \left((Q\tilde{S}^T-\tilde{S}Q^T)^\vee\right)^T} \tilde{U}^T
\end{align}
\begin{align}
	\expect{x(0)\nu'^T_{R(0)(\hat{b}_g(0)-\hat{\mu}_{b_g})\delta R}} &= \expect{x(0)\left( b_g^T(0) - \mu^T_{b_g} \right)V\Gamma_Q^T} \tilde{U}^T \nonumber \\
	&= \expect{\left( \Sigma_{c,xb_g} + \mu\nu^T_RP_{b_g}^T + P\nu_R\nu_R^TP_{b_g}^T\right)V\Gamma_Q^T} \tilde{U}^T
\end{align}
\begin{align}
	\expect{x(0)\nu'^T_{R(0)((H_{gu}\Delta W_{gu})^\wedge)^2\delta R}} = h\expect{x(0)\nu'^T_{R(0)G_{gu}\delta R}} - \tr{G_{gu}}h\expect{x(0)\nu'^T_{R(0)\delta R}}.
\end{align}

\section{Measurement Update}

Suppose $(R,x)$ follows a matrix Fisher distribution with parameters $(\mu,\Sigma,P,U,S,V)$, and a measurement of $x$ follows a Gaussian distribution with parameters $(\mu_m,\Sigma_m)$, then the conditional posterior distribution of $x$ is
\begin{align}
	&\expb{-\frac{1}{2}(x-\mu_c)^T\Sigma_c^{-1}(x-\mu_c)} \cdot \expb{-\frac{1}{2}(x-\mu_m)^T\Sigma_m^{-1}(x-\mu_m)} \nonumber \\
	\propto \ &\expb{-\frac{1}{2} \left( x^T\left(\Sigma_c^{-1}+\Sigma_m^{-1}\right)x - 2x^T\left(\Sigma_c^{-1}\mu_c+\Sigma_m^{-1}\mu_m\right) + \mu_c^T\Sigma_c^{-1}\mu_c \right)} \nonumber \\
	= \ &\expb{-\frac{1}{2}\left(x-\left(\Sigma_c^{-1}+\Sigma_m^{-1}\right)^{-1}\left(\Sigma_c^{-1}\mu_c+\Sigma_m^{-1}\mu_m\right)\right)^T (\Sigma_c^{-1}+\Sigma_m^{-1}) \left(x-\left(\Sigma_c^{-1}+\Sigma_m^{-1}\right)^{-1}\left(\Sigma_c^{-1}\mu_c+\Sigma_m^{-1}\mu_m\right)\right)} \nonumber \\
	\quad &\times \expb{\frac{1}{2} \left(\left(\Sigma_c^{-1}+\Sigma_m^{-1}\right)^{-1}\left(\Sigma_c^{-1}\mu_c+\Sigma_m^{-1}\mu_m\right)\right)^T (\Sigma_c^{-1}+\Sigma_m^{-1}) \left(\left(\Sigma_c^{-1}+\Sigma_m^{-1}\right)^{-1}\left(\Sigma_c^{-1}\mu_c+\Sigma_m^{-1}\mu_m\right)\right) - \frac{1}{2}\mu_c^T\Sigma_c^{-1}\mu_c } \nonumber \\
	\propto \  &\expb{-\frac{1}{2}\left(x-\left(\Sigma_c^{-1}+\Sigma_m^{-1}\right)^{-1}\left(\Sigma_c^{-1}\mu_c+\Sigma_m^{-1}\mu_m\right)\right)^T (\Sigma_c^{-1}+\Sigma_m^{-1}) \left(x-\left(\Sigma_c^{-1}+\Sigma_m^{-1}\right)^{-1}\left(\Sigma_c^{-1}\mu_c+\Sigma_m^{-1}\mu_m\right)\right)} \nonumber \\
	\quad &\times \expb{-\frac{1}{2}(\mu_c-\mu_m)^T(\Sigma_c+\Sigma_m)^{-1}(\mu_c-\mu_m)}
\end{align}
Note that only the first exponential term has $x$, however, the second exponential term has $\nu_R$ in $\mu_c$, therefore it cannot be discarded as a constant.
Denote $\expb{f(R)}$ as the second exponential term, then the posterior marginal 
distribution of $R$ becomes
\begin{equation} \label{eqn:postAttitude}
	\expb{\tr{FR^T}+f(R)}.
\end{equation}
This density is a matrix Fisher-Bingham distribution on $\SO$, since $f(R)$ clearly involves the second order term of $R$.
Intuitively, if we think of the matrix Fisher distribution as a Gaussian distribution in $\real{9}$, then the measurement of $x$ makes some correction to $R$, thus its mean is no longer in $\SO$, and conditioning the corrected Gaussian distribution no longer yields a matrix Fisher distribution.
After omitting some constants, \eqref{eqn:postAttitude} can be simplified as

\begin{align}
	\expb{\tr{FR^T}+f(R)} \propto \expb{\tr{F^TR} - \nu_R^TP^T(\Sigma_c+\Sigma_m)^{-1}(\mu-\mu_m) - \frac{1}{2}\nu_R^TP^T(\Sigma_c+\Sigma_m)^{-1}P\nu_R}.
\end{align}

Let $x = [x_1,x_2]$, suppose only the measurement of $x_1$ is available, which follows a Gaussian distribution with parameters $(\mu_m,\Sigma_m)$.
\end{document}

