\documentclass[10pt]{article}

\usepackage{amssymb,amsmath,amsthm}
\usepackage{bm}
\usepackage{graphicx,subcaption}
\usepackage[letterpaper, top=1in, left=1in, right=1in, bottom=1in]{geometry}

\newtheorem{definition}{Definition}
\newtheorem{theorem}{Theorem}
\newtheorem{lemma}{Lemma}
\newtheorem{remark}{Remark}

\title{\vspace{-4ex}\textbf{$\mathrm{SO}(3)\times\mathbb{R}^n$ Expressed as a Matrix Lie Group\vspace{-4ex}}}
\date{}

\graphicspath{{./figs/}}

\begin{document}

\maketitle

Consider the following map:
\begin{equation*}
	\Phi:\,\mathrm{SO}(3)\times\mathbb{R}^n\to GL(3+2n,\mathbb{R}), \qquad
	\Phi(R,x) = \begin{bmatrix}
		R & 0 & 0 \\
		0 & I & \mathrm{diag}(x) \\
		0 & 0 & I
	\end{bmatrix}
\end{equation*}
$\Phi$ is clearly one-to-one, and it preserves the group multiplication and inverse as follows
\begin{gather*}
	\Phi(R_1R_2,x_1+x_2) = \begin{bmatrix}
		R_1R_2 & 0 & 0 \\
		0 & I & \mathrm{diag}(x_1+x_2) \\
		0 & 0 & I
	\end{bmatrix} = \\
	\begin{bmatrix}
	R_1 & 0 & 0 \\
	0 & I & \mathrm{diag}(x_1) \\
	0 & 0 & I
	\end{bmatrix}\begin{bmatrix}
	R_2 & 0 & 0 \\
	0 & I & \mathrm{diag}(x_2) \\
	0 & 0 & I
	\end{bmatrix} = \Phi(R_1,x_1)\Phi(R_2,x_2)
\end{gather*}
and
\begin{equation*}
	\Phi(R^T,-x) = \begin{bmatrix}
		R^T & 0 & 0 \\
		0 & I & -\mathrm{diag}(x) \\
		0 & 0 & I
	\end{bmatrix} = \begin{bmatrix}
	R & 0 & 0 \\
	0 & I & \mathrm{diag}(x) \\
	0 & 0 & I
	\end{bmatrix}^{-1} = \Phi(R,x)^{-1}.
\end{equation*}
So $\Phi$ is a group isomorphism from $\mathrm{SO}(3)\times\mathbb{R}^n$ to a subgroup of $GL(3+2n,\mathbb{R})$.
Since SO(3) is closed in $GL(3,\mathbb{R})$, and clearly the limit of a convergent sequence in subset $\left\{\left.\begin{bmatrix}
	I & x \\
	0 & I
\end{bmatrix}\right|x\in\mathbb{R}^n\right\}\subset GL(2n,\mathbb{R})$ still lies in the same subset.
Therefore, the image of $\Phi$ is a closed subgroup in $GL(3+2n,\mathbb{R})$, which is by definition a matrix Lie group.
Furthermore, $\Phi$ and $\Phi^{-1}$ are clearly smooth, so $\Phi$ is a Lie group isomorphism.

\end{document}

