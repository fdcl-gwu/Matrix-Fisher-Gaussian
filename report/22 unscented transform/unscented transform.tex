\documentclass[10pt]{article}

\usepackage[letterpaper,margin=0.5in]{geometry}

\usepackage{amssymb,amsmath,amsthm}
\usepackage{bm}
\usepackage{graphicx,subcaption}
\usepackage{xcolor}

\newtheorem{definition}{Definition}
\newtheorem{proposition}{Proposition}
\newtheorem{theorem}{Theorem}
\newtheorem{lemma}{Lemma}
\newtheorem{remark}{Remark}

\title{\vspace{-4ex}\textbf{Unscented Transform for Matrix Fisher Distribution\vspace{-4ex}}}
\date{}

\newcommand{\norm}[1]{\ensuremath{\left\| #1 \right\|}}
\newcommand{\fnorm}[1]{\ensuremath{\left\| #1 \right\|_\mathrm{F}}}
\newcommand{\tr}[1]{\ensuremath{\mathrm{tr}\left( #1 \right)}}
\newcommand{\etr}[1]{\ensuremath{\mathrm{etr}\left\{ #1 \right\}}}
\newcommand{\expect}[1]{\ensuremath{\mathrm{E}\left[ #1 \right]}}
\newcommand{\SO}{\ensuremath{\mathrm{SO(3)}}}
\newcommand{\real}[1]{\ensuremath{\mathbb{R}^{ #1 }}}
\newcommand{\diff}[1]{\ensuremath{\mathrm{d} #1}}
\newcommand{\expb}[1]{\ensuremath{\mathrm{exp}\left\{#1\right\}}}

\begin{document}

\maketitle

Suppose $R\sim \mathcal{M}(S)$, we want to select sigma points and weights $(R_i,w_i)$ from the distribution, such that $\expect{R} = \sum w_iR_i$.
Let $R_i = I_{3\times 3}$, $\exp(\pm\theta_1\hat{e}_1)$, $\exp(\pm\theta_2\hat{e}_2)$, $\exp(\pm\theta_3\hat{e}_3)$, and their weights be $w_0$, $w_1$, $w_2$, $w_3$.
Then we have
\begin{align}
	\sum w_iR_i = \mathrm{diag} \begin{bmatrix}
		w_0 + 2w_1 + 2w_2\cos\theta_2 + 2w_3\cos\theta_3 \\
		w_0 + 2w_1\cos\theta_1 + 2w_2 + 2w_3\cos\theta_3 \\
		w_0 + 2w_1\cos\theta_1 + 2w_2\cos\theta_2 + 2w_3
	\end{bmatrix} = \mathrm{diag} \begin{bmatrix} d_1 \\ d_2 \\ d_3 \end{bmatrix}.
\end{align}
For $d_1+d_2-d_3$, $d_1+d_3-d_2$, and $d_2+d_3-d_1$, we have
\begin{equation}
	\begin{aligned}
		1 + 4w_3(\cos\theta_3-1) &= d_1+d_2-d_3, \\
		1 + 4w_2(\cos\theta_2-1) &= d_1+d_3-d_2, \\
		1 + 4w_1(\cos\theta_1-1) &= d_2+d_3-d_1.
	\end{aligned}
\end{equation}
Thus, $\theta_i$ can be solved as
\begin{align} \label{eqn:theta}
	\theta_i = \arccos\left( 1 - \frac{1+d_i-d_j-d_k}{4w_i} \right),
\end{align}
for $\{i,j,k\} = \{1,2,3\}$.
In any case, we want $\theta_i < \frac{2\pi}{3}$, this indicates $w_i \geq \tfrac{1+d_i-d_j-d_k}{6}$, so
\begin{align}
	w_0 = 1-2(w_1+w_2+w_3) \leq \tfrac{1}{3}(d_1+d_2+d_3).
\end{align}
For a parameter $\lambda \in [0,1]$, we can design the weights to be
\begin{equation}
	\begin{aligned}
		w_0 &= \tfrac{\lambda}{3}(d_1+d_2+d_3), \\
		w_i &= \tfrac{1}{6}(1+d_i-d_j-d_k) + \tfrac{1-\lambda}{18}(d_1+d_2+d_3).
	\end{aligned}
\end{equation}

\end{document}

